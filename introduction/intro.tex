\begin{document}
	\chapter{Introduction}
		One of the most significant fact that occurred during 2017 was the boom of the cryptocurrencies. In September 2017 the networks were flooded with news about a virtual currency that was rapidly gaining notoriety (and thus, value) for being a "decentralized" currency, that is, a monetary system that doesn't need the intermediation of a trusted third party (i.e a bank, a government) to be issued or to be spent. The name was Bitcoin.
		
		Bitcoin and other cryptocurrencies were overwhelmed by a market frenzy that caused the money's price to sky-rocket from 3'500 \$ each to nearly 20'000 \$\footnote{http://fortune.com/2017/12/17/bitcoin-record-high-short-of-20000/} by the end of December.
		The rise of interest in the cryptocurrency world has not only concerned investors and/or speculators but also has led many developer and researchers to work on the subject: by the end of 2017 the number of publications strictly related to Bitcoin was 664, doubling up the number of publications in 2016\footnote{https://cdecker.github.io/btcresearch/}. However, it is in that occasion that the limitations of the Bitcoin protocol emerged: slow transactions processing, high processing fees and ever increasing ledger size posed new threats to a system that has faced such a huge popularity.
		
		By the time of writing, the most notorious cryptocurrencies are currently facing the consequences of such an awesome ride, having lost in the first quarter of the year most of their value (Bitcoin itself dropped from 20'000 \$ back to 7'000 \$) due to speculation and market manipulation performed by some big players.  Nevertheless, the researchers are more interested on the technology behind Bitcoin and every other cryptocurrency, which is the \textbf{blockchain}, the new disruptive technology that is posing new challenges and new horizons to decentralized systems. 
		
		A large part of the researchers community is focusing on the scalability problem of the \textit{proof-of-work}-based blockchains (e.g. Bitcoin, Ethereum, Litecoin, Bitcoin Cash...) [citation needed]: the throughput of transactions processed by the current Bitcoin protocol is 7 transactions per second, far below the throughput of the current dominant transaction processors like VISA average of 2'000 transactions per second (up to 65'000 transactions per second\footnote{https://usa.visa.com/dam/VCOM/download/corporate/media/visanet-technology/aboutvisafactsheet.pdf}). Addressing this problem has been the goal of the blockchain community since the very first time the blockchain technology was born. There have been improvements over the year (SegWit update or the block size increase update), but none of these solution came even nearly close to the transactions per second processed by their main competitor.

		On January 2016, Joseph Poon and Thaddeus Dryja published a white paper describing a layer 2 technology called Lightning Network\footnote{https://lightning.network/}. The idea behind it is very simple: a network of micro-payments channel (i.e. transaction channels between two peers) where each Bitcoin transfer occurs off-blockchain without having trust issues. The idea was instantly seen as something revolutionary among the members of the blockchain community, and as soon as the prototype network went live on the Bitcoin testnet (a testing environment for developers) thousands of users joined it. In late 2017 the Lightning Network faced a new wave of popularity thanks to the golden period Bitcoin was facing and this accelerated the developing process of the network that lead to the 15th of March 2018 release, the day where the Lightning Network went live on the Bitcoin main-net. This pre-release (which is still in its beta version) is addressed to developers in order to get them ready with a plethora of Lightning Apps when the Lightning Network will officially come out from the beta release.
		
		The aim of the project presented in this thesis is to provide an analysis of said network, measure its robustness using ad-hoc metrics and provide a characterization that would allow to build an equivalent model in order to study the network in a more wider scenario. The project will be structured as follows:
		\begin{itemize}
			\item The second chapter will give a \text{background} of the blockchain technology, the Bitcoin protocol and the Lightning Network.
			\item The third chapter will describe how the network data is collected and all the tools used to produce the knowledge base for the analysis.
			\item Lastly, the last chapter will visualize and discuss the results and the implication of the analysis of the data collected.
		\end{itemize}
	\section{Motivation}
		Decentralized payment networks and decentralized ledgers are a new entity in IT world and their success relies on guarantees on fairness, privacy, security, availability and persistence: a payment network that is not compliant with these properties is a network that should not to be trusted in any way. The Lightning Network leverage the properties of the Bitcoin blockchain to ensure that untrusted payment routes can be performed, yet it does so by adding a new complex layer on top of the Bitcoin architecture. This layer forms a network whose topology resembles the topology of a \textit{scale-free network} [citation needed]. It is of interest to analyze this network and understand the properties that are provided to the user. 
		
		There is also an ongoing debate over the scalability problem (that will be discussed in details in the next chapter) whether or not is it better to add a layer of complexity instead of increasing the blocksize of each block. This debate escalated and lead to what is known to be a \textit{hard-fork}, that is a  permanent split in the history of the ledger that divided the Bitcoin community into Bitcoin and Bitcoin Cash.
\end{document}