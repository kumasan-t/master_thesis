\documentclass[LaM, oneside, english, noexaminfo]{sapthesis}
\usepackage{subfiles}
\usepackage{setspace}
\usepackage{graphicx}
\usepackage{tabulary}
\usepackage{listings}
\usepackage{subcaption}
\usepackage{url}
\usepackage{amsmath}
\usepackage{amsfonts}
\usepackage{xcolor}
\usepackage{makecell }
\usepackage{amsthm}

\theoremstyle{definition}
\newtheorem{definition}{Definition}
\definecolor{backgroundgray}{RGB}{239, 239, 239}
\definecolor{weborange}{RGB}{255,165,0}

%\setlength{\parindent}{4em}

\graphicspath{ {./images/} }

\title{Bitcoin Lightning Network Payment System: Network Structure Characterization}
\author{Lorenzo Travagliati}
\IDnumber{1495035}
\course[Faculty of Information Engineering Computer Science and Statistics]{Engineering in Computer Science}
\courseorganizer{Faculty of Information Engineering, Computer Science and Statistics}
\submitdate{2017/2018}
\copyyear{2018}
\advisor{Prof.ssa Silvia Bonomi}
\authoremail{1495035.travagliati@studenti.uniroma1.it}

\onehalfspacing
\makeindex

\begin{document}
	
	\frontmatter
	\maketitle
	
	\dedication{Dedicato a\\ i miei nonni.}
	\begin{abstract}
	The Bitcoin protocol as it is has several scalability issues: low transaction processing time, ever-growing blockchain size and high transaction fees make it impossible for users to perform small payments over the network. For this reason, a layer 2 protocol known as Lightning Network has been proposed and recently a beta implementation has been deployed on the Bitcoin testing environment. Rapidly the Lightning Network has grown in size, with hundreds of clients joining the network to test out this disruptive evolution of the cryptocurrencies and soon a newer, stable version has been published allowing for payments to be performed in the main-net environment.
	
	We are interested in the characterization of such system, putting in evidence its core aspects by providing a trend reporting analysis over some features of the network itself. Then we investigate its topological structure and how these findings can be leveraged to stress further the decentralized aspect of the system. Next, we discuss which random graph models are best suited to build an equivalent graph which may help to study the natural evolution of the network as more and more users decide to participate. Lastly we provide a formalization of the network dynamics based on a recent framework known as Time Varying Graph that captures the highly dynamic behavior of the network components.
	\end{abstract}
	
	\tableofcontents
	\mainmatter
	\setlength{\parskip}{1em}
	\subfile{introduction/intro}
	\subfile{background/background}
	\subfile{dynamic_networks/dynamic_networks}
	\subfile{material_methods/material_methods}
	\subfile{results/results}
	\subfile{conclusion/conclusion}
	\bibliographystyle{plain}
	\bibliography{bibliography.bib}
	
\end{document}
