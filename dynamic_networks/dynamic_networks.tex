\begin{document}
	\chapter{Dynamic Networks}
	The strict relationship between time and money in the Lightning Network puts it in an incredibly interesting spot regarding the dynamic properties of it: there must exist channels connecting pair of nodes along a payment path and every node involved in the payment path must be online in the moment a transaction is performed. A third constraint on a payment procedure is that every node involved in the payment process must have enough funds to transfer.
	
	The three constraints play a key role in the modeling of the network and put in evidence the dynamic feature of the network itself. Therefore the network is not a static entity but a dynamic one, and a lots of its feature will be shown have similarities with social networks. To proceed further in the analysis of the Lightning Network it is necessary to introduce the concept of Time Varying Graphs.
	
	Communication networks, transportation systems, social studies, biology and neuroscience all presents some characterization element that can be related to a social network. Social network are structures were social actors interact with each other and through the \textit{social network analysis} it is possible to undergo deeply on the characterization of the network itself. While social network emerged as a prominent sociology topic in 1908 by Georg Simmel, it was later re-discovered by physicists and mathematically formalized in late 1950s\cite{Simon1957}. Through the years many researchers developed newer and more fine-grained models for these networks (Barabási–Albert\cite{Barabasi1999}, Erdős–Rényi\cite{Erdos1959}, Watts–Strogatz\cite{Watts1998}) but although these models capture many common features, they lack in the description of the dynamic behavior which is crucial in some models. 
	
	The dynamicity of a network has lately seen intensive research efforts and a vast literature is available, yet it often results to be very problem-specific and lacks of a universal formal description. An important contribution to this research area was the formalization of \textit{Time-Varying Graphs}\cite{Casteigts2012} that is a unified framework that address the main characteristics of a dynamic network, and whose goal is to put in evidence and formally define important concepts that were present in other research areas (delay-tolerant networks, opportunistic networks, real-world complex networks) but not related each other.
	
	\section{Time Varying Graph}

	A Time-Varying Graph (TVG) is described by a quintuple \(\text{TVG} = (V, E, T, \rho, \zeta) \):
	\begin{itemize}
		\item a set of nodes \(V\).
		
		\item a set of relations between nodes \(E\) (edges).	
		
		\item an alphabet \(L\) (optional).
		
		\item a relation \(E \subseteq V \times V \times L\) where \(L\) are domain-specific labels and can be used (or omitted) to enrich the description of the network.
		
		\item a time span *simbolo mancante* \( T \subseteq \mathbb{T}\) called \textit{lifespan} with \(\mathbb{T}\) representing the domain of time which usually coincides with the \(\mathbb{N}\) if the system is discrete, \(\mathbb{R}^+\) otherwise.
		
		\item a function \(\rho : E \times T \to \{0, 1\} \) called \textit{presence} to track the availability of an edge at a given time.
		
		\item a function \(\zeta : E \times T \to \mathbb{T}\) called latency function that indicates the time needed to cross an edge at a given date.
	\end{itemize}

	The model can also be enriched with two more functions that can capture the nodes dynamic behavior in a way similar to \(\rho\) and \(\zeta\).
	\begin{itemize}
		\item a function \(\psi : V \times T \to \{0, 1\}\) called \textit{node presence function}, expressing the availability of a node at a given time T.
		
		\item a function \(\phi : V \times T \to \mathbb{T}\) called 
	\end{itemize}
	
	Inside the definition of a time varying graph we have the notion of \textit{underlying graph}, that is the graph \(G = (V, E)\) which is the backbone of the dynamic network. One important thing to notice is that a connected component G doesn't imply the connectivity of the TVG, in fact the TVG could be disconnected for every time instant of its lifespan.
	
	\begin{figure}
		\begin{subfigure}{0.5\textwidth}
			\centering
			\includegraphics[scale=0.65]{example_tvg}
			\caption{The visual representation of a TVG.}
		\end{subfigure}
		\begin{subfigure}{0.5\textwidth}
			\centering
			\includegraphics[scale=0.65]{example_underlying}
			\caption{The underlying graph.}
		\end{subfigure}
		\begin{subfigure}{0.5\textwidth}
			\centering
			\includegraphics[scale=0.65]{example_tvg_0_1}
			\caption{Edges availability according to presence function at time interval [0, 1].}
		\end{subfigure}
		\begin{subfigure}{0.5\textwidth}
			\centering
			\includegraphics[scale=0.65]{example_tvg_2_3}
			\caption{Edges availability according to presence function at time interval [2, 3].}
		\end{subfigure}
		\caption{Layers of a Time Varying Graph. Image courtesy of \cite{Casteigts2012}.}
	\end{figure}
	
	\begin{itemize}
		\item \textit{Edge-centric evolution}: every edges has associated a set of the union of all the dates in which an edge \(e\) is available. Such set is denoted by \(I(e)\) and every element of \(I\) is such that \(I(e) = \{t \in T | \rho(e,t) = 1\}\), that is, the elements of the set are of the kind \(I(e) = \{t_1, t_2, t_3 ... \}\) and since they express time intervals it is possible to subdivide them in two sets \(App(e)\) and \(Dis(e)\), \textit{appearance and disappearance}, where \(App(e) = \{t_n | t_n \in I(e), n = 2N + 1\}\) and \(Dis(e) = \{t_n | t_n \in I(e), n = 2N\}\), i.e. every \(t\) in an even position is a date of appearance while \(t\) in odd position are dates of disappearance.
		
		\item \textit{Vertex-centric evolution}: nodes spawn and de-spawn causes changes in the configuration of the neighbor for some nodes, and it is expressed through \textit{sequences of neighborhoods} \(N_{t_1}(v), N_{t_2}(v), N_{t_3}(v) ...\). This kind of configuration is not very common in the literature.
		
		\item \textit{Graph-centric evolution}: it is the situation in which graphs are subject to events (edges appearance and disappearance) that make their topology change. Each of this events happens in a \textit{characteristic date}, and these dates are sorted in chronological such that \(S_{T}(\overline{G}) = sort(\cup\{S_t(e) : e \in E\})\) with \( \overline{G}\) being a TVG. From a global pont of view, it is possible to have a look at the evolution of the graph as a sequence of graph snapshot \(S_{\overline{G}} = G_1, G_2, G_3\) where \(G_i\) is a static snapshot of \(\overline{G}\) at time \(i\).
	\end{itemize}
	
	The time-varying graph framework also put emphasis on the different peculiarity of the different networks that can be modeled as a TVG by providing a classification method based on some particular properties. They are thirteen in total and are \textit{minimum reachability, all-nodes-reachability, connectivity over time, round connectivity, recurrent connectivity, recurrence of edges, time bounded recurrence of edges, periodicity of edges, constant connectivity, T-interval connectivity, eventual instant-connectivity, eventual instant-routability, complete graph of interaction}.
	Such a classification should help to categorize better a problem, and to help navigating the literature with well defined properties that have been investigated in other works.
\end{document}